\section{Setting the Coordinate List with {\tt -cl}}
Recall that a Map is composed of layers, and that each layer is an
array of elements. Therefore, each element corresponds to a
coordinate: the layer index and the element index within the
layer. For example, we refer to element 3 in layer 1 by the coordinate
(1,3). All of the indices are zero-based. To compute the output of a
map, the user defines which elements are outputs, and in which order
they will appear.

This is done on the command line using the argument {\tt -cl},
followed by a coorinate list definition string. The format of this
string is described in detail in this section.

To simplify our exposition, we use the following example: say that our
map has three layers, layer 0 has 5 elements, layer 1 has 4 elements,
and layer 2 has 2 elements.

The definition string is a semicolon (;) separated list of individual
definitions. The coordinate list generated by the entire string is the
concatenation of the lists generated by each of the individual
definitions. So, if the string ``{\tt 1}'' generates the list
$$
\texttt{1} \quad\rightarrow\quad (1,0) (1,1) (1,2) (1,3)
$$
and the string ``{\tt 2}'' generates the list
$$
\texttt{2} \quad\rightarrow\quad (2,0) (2,1)
$$
then the string ``{\tt 1;2}'' generates the concatenated list 
$$
\texttt{1;2} \quad\rightarrow\quad (1,0) (1,1) (1,2) (1,3) (2,0) (2,1)~~.
$$ 
%

\paragraph{Generating All of the Coordinates in a Layer}
If the individual definition is a single non-negative integer, it
generates all of the coordinates in a corresponsing layer, in
order. This explains the example above. Alternatively, the letter
``{\tt e}'' (stands for \emph{end}) can be used to represent the
maximal possible index value. Therefore, in our example, the string
``{\tt e}'' generates coordinates in the last layer, namely, 
$$
\texttt{e} \quad\rightarrow\quad (2,0) (2,1)~~.
$$ 
%
Layers can also be referenced from the end, using the syntax ``{\tt
  e-<uint>} `` where {\tt <uint>} is a replaced by a non-negative
integer. For example, ``{\tt e-1}'' the layer before last. In our
example
%
$$
\texttt{e-1} \quad\rightarrow\quad (1,0) (1,1) (1,2) (1,3)~~.
$$ 

\paragraph{Generating a Specific Element}
To specify a specific element, use the comma (,) character after the
layer index, as in
%
$$
\texttt{1,3} \quad\rightarrow\quad (1,3)~~.
$$ 
%
Again, ``{\tt e}'' and ``{\tt e-<uint>}'' can be used to reference
the elements from the end, as in 
\begin{eqnarray*} 
\texttt{2,e}&\quad\rightarrow\quad&(2,1)\\
\texttt{2,e-1}&\quad\rightarrow\quad&(2,0)
\end{eqnarray*} 

\paragraph{Generating an Interval of Elements in a Layer}
To specify an interval of elements in a single layer, use the colon
(:) character, as in the following examples
%
$$
\texttt{1,1:3} \quad\rightarrow\quad (1,1) (1,2) (1,3)~~.
$$ 
%
Again, ``{\tt e}'' and ``{\tt e-<uint>}'' can be used to enumerate the
elements from the end. 

\paragraph{Examples}
We conclude with a few examples
\begin{eqnarray*} 
\texttt{e,e;0}&\quad\rightarrow\quad&(2,1)(0,0)(0,1)(0,2)(0,3)(0,4) \\
\texttt{1,e-2:e-1;e-1,2}&\quad\rightarrow\quad&(1,2)(1,3)(1,2) \\
\texttt{e;e,e:e;e,e}&\quad\rightarrow\quad&(2,0)(2,1)(2,1)(2,1)
\end{eqnarray*} 



Formally, the coordinate list definition string follow the
following set of regular expression rules.
\begin{eqnarray*}
\texttt{<defString>} &\quad\Rightarrow\quad& \texttt{<def>;<defString>}\\
\texttt{<def>} &\quad\Rightarrow\quad& \texttt{<layerIndex>}\\
&&\texttt{| <layerIndex>,<elementInterval>}\\
&&\texttt{| <layerIndex>,<elementIndex>}\\
\texttt{<layerIndex>} &\quad\Rightarrow\quad& \texttt{<index>}\\
\texttt{<elementInterval>} &\quad\Rightarrow\quad& \texttt{<index>:<index>}\\
\texttt{<elementIndex>} &\quad\Rightarrow\quad& \texttt{<index>}\\
\texttt{<index>} &\quad\Rightarrow\quad& \texttt{<uint>}\\
&&\texttt{| e}\\
&&\texttt{| e-<uint>}
\end{eqnarray*}
